\subsection{Host software}
The remote software is either implemented on a PC or on a server. It is used for performing the training of the network and for generating a FPGA-bitstream based on the computed weights. Additionally the remote software is used to send the image data to the Zedboard and receive the results of the network for each image. \\
Therefore the Host software can be separated in two parts:
\begin{itemize}
	\item Trainings software
	\item Communication software
\end{itemize} 
Requirements of the Trainings Software:
\begin{itemize} 
	\item Training of the network considering bit resolution of implemented hardware
	\item Create VHDL code based on the network hyper-parameter and on the computed weights
	\item Create a bitstream with the generated vhdl code
\end{itemize}
Requirements of the Trainings Software:
\begin{itemize}
	\item Sends image data to Zedboard
	\item Receives results from Zedboard
	\item Create a figure of accuracy and performance   
	\item Optional: Send bitstream to hardware which updates the bitstream 
\end{itemize}
\subsubsection{Interface to Zedboard} \label{subsec:InterfaceRemoteZed}
Ethernet is used for the communication of the remote host system and the embedded Linux which is running on the Zedboard.\\ 
The embedded Linux distribution running on the board should automatically receive an IP address when connected to a network. When in doubt the address can be found out with the \verb|ifconfig| command. \\
The software has a client-server model with the embedded system acting as a server and the host as a client. Once running, the server software is listening for new outside connections.\\ \todo{Who is the host and client now?}
Different types of data need to be transmitted:
\begin{itemize}
	\item The 28x28 input images showing digits between 0 and 9 is transferred from host to Zedboard.
	\item The probability of resulting numbers between 0 and 9 is transmitted from Zedboard to host.
	\item control and status signals in both directions 
	\item Optional: Bitstream file for dynamically update the bitstream at the Zedboard
\end{itemize}
\subsubsection{Notes}
On Windows host systems, \emph{Network Discovery} needs to be enabled and in some cases a Firewall exception for the used ports needs to be set for a connection to be established. \\