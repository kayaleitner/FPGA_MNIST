\subsection{Remote software}

It is used for performing the training of the network and for generating a FPGA-bitstream based on the computed weights. Additionally the remote software is used to send the image data to the Zedboard and receive the results of the network for each image. 

Therefore the Host software can be separated in two parts:
\begin{itemize}
	\item Trainings software
	\item Communication software
\end{itemize} 

Requirements of the Trainings Software:
\begin{itemize} 
	\item Training of the network considering bit resolution of implemented hardware
	\item Create VHDL code based on the network hyper-parameter and on the computed weights
	\item Create a bitstream with the generated VHDL code
\end{itemize}

Requirements of the Trainings Software:
\begin{itemize}
	\item Sends image data to Zedboard
	\item Receives results from Zedboard
	\item Create a figure of accuracy and performance   
	\item Optional: Send bitstream to hardware which updates the bitstream 
\end{itemize}

\subsubsection{Interface to Zedboard} \label{subsec:InterfaceRemoteZed}
As interface to the zedboard a flask application is running on the zedboard which allows to upload a new image of a digit, test the hardware accelerate using the MNIST-dataset, gives useful information and allows to update the FPGA bitstream. 

\begin{itemize}
	\item The uploaded images are resized to $28 \times 28$ automatically
	\item The MNIST-dataset can be used
	\item Status information about the inference time and CPU utilization are shown
	\item Bitstream file for update the hardware accelerator can be uploaded. 
\end{itemize}