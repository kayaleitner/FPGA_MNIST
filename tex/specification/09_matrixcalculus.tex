\subsection{Matrix Calculus}

The chain rule for a vectors is similar to the chain rule for scalars. Except the order is important. For $\mathbf{z} = f(\mathbf{y})$ and $\mathbf{y} = g(\mathbf{x}) $ the chain rule is:
\begin{equation}
    \frac{\partial \mathbf{z}}{\partial \mathbf{x}} = \frac{\partial \mathbf{y}}{\partial \mathbf{x}}     \frac{\partial \mathbf{z}}{\partial \mathbf{y}}
\end{equation}




\begin{table}[h]
    \centering
    \begin{tabular}{cc}
        \toprule
            $y$ & $\frac{\partial}{\partial x} y$ \\
        \midrule
            $Ax$     & $A^T$ \\
            $x^T A$  & $A$   \\
            $x^T x$  & $2x$  \\  
            $x^T Ax$ & $Ax + A^Tx$  \\          
        \bottomrule
    \end{tabular}
    \caption{Useful derivatives equations}
\end{table}


\subsection{Fix-Point Arithmetic}

Multiplication of two fix point values yields 
\begin{equation}
	v_1 v_2 = \text{right-shift}\left( Q_1 Q_2 \cdot 2^{-(m+m)} ; m\right)
\end{equation}
Note that for multiplication the exponent $m$ for the values can be different.

Addition of two fix point values
\begin{equation}
	v_1 + v_2 = \left( Q_1 + Q_2 \right) \cdot 2^{-m}
\end{equation}